This section details the safety requirements to the bluetooth hydrometer use. The risk factors of the hydrometer being a low-voltage electrical device is that it may pose a slight electrocution risk to the end-user, and a short-circuit may damage the electronic components of the hydrometer including the main Arduino board.

\subsection{Requirement Name}
\subsubsection{Description}
As the bluetooth hydrometer is going to be a device that is placed in liquid for use, it is important to ensure that the outer capsule is kept closed and sealed to the entirety of the time it is used. The user must ensure that he/she is grounded before working with the hydrometer.
\subsubsection{Source}
This is a basic project safety requirement to make sure that the user does not get electric shocked and that the device does not become short circuited.
\subsubsection{Constraints}
The user will be handling the Arduino board directly as it has no other protective covering when outside of the outer capsule.
\subsubsection{Standards}
Not applicable.
\subsubsection{Priority}
This is of critical importance for device use.